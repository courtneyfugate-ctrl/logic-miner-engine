\documentclass{article}
\usepackage[utf8]{inputenc}
\usepackage{reledmac}
\usepackage{amsmath}
\begin{document}
\title{Systematic Organization of Animalia}
\maketitle
\section*{P-adic Logic Analysis}
\beginnumbering
\pstart
The analysis yielded a \textbf{Semantic Ultrametric Tree}. \\
Tree Structure (Topology variants): \\
\texttt{((Whale_{v_p=2})(Despite_{v_p=2})((Bone_{v_p=3})((Single_{v_p=1},Jaw_{v_p=1}))((((Mammalia_{v_p=4})Mammalian_{v_p=2},Platypus_{v_p=2}))(((Animal_{v_p=10})((Innovation_{v_p=2})(Integration_{v_p=2})(Bones_{v_p=2})(Ear_{v_p=3})(((((Mammals_{v_p=4},Feature_{v_p=5}),Key_{v_p=3}),(Mammary_{v_p=4},Glands_{v_p=4})))(((Class_{v_p=5})(Kingdom_{v_p=6})(Taxonomy_{v_p=5})(Animalia_{v_p=6})(Part_{v_p=7})Phylum_{v_p=4})Their_{v_p=2})Chordata_{v_p=2},Hair_{v_p=1}),Order_{v_p=1}),Logic_{v_p=1}),Cnidaria_{v_p=1}))(Three_{v_p=1},Middle_{v_p=1}),Its_{v_p=1}),However_{v_p=1});}
\pend
\endnumbering
\end{document}